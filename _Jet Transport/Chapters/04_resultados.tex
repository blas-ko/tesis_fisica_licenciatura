En el capítulo \ref{chap:CR3BP} se estudió el problema restringido circular de los tres cuerpos. Dos partículas primarias orbitando en un plano a velocidad constante respecto a su centro de masa y una tercera influenciada por las otras dos con una dinámica muy rica. Rica es por elementos como la sensibilidad al parámetro de masa (\ref{eq:3body_mu}), la sensibilidad a condiciones inciales cercanas, las trayectorias discriminadas por su constante de Jacobi, o los cinco puntos lagrangianos (\ref{eq:3body_lag}). Esta sección busca explotar el transporte de jets como se describe en el capítulo \ref{sec:jt} así como los indicadores y herramientas desarrolladas en \ref{sec:jt_indicators}.

En este capítulo se buscan explotar las propiedades dinámicas encontradas en \ref{chap:CR3BP} con ayuda de las herramientas planteadas en \ref{chap:jt_indicators}. Se utilizará al tranporte de jets en varias perspectivas donde, por ejemplo, se lo tratará como un integrador de flujo cuyas condiciones iniciales no se saben con presición, una herramienta para encontrar las ecuaciones de primera variación y más, método para encontrar las máximas y mínimas expansiones de la vecindad de una condición inicial dada en el espacio de configuraciones

Este capítulo se divide como sigue: 
%...

%Algo con las 0VC, 
%Parametrización de la masa (no está jalando; chance lo mochamos)
%heatmaps..
%direcciones de mayor y menor contracción alrededor de un punto
%simplecticidad??????
%ξmax en algo que no sea ξgrid? 

\section{Parametrización de $\mu$}
Como se analizó en la sección \ref{sec:lag_points}, $L_i = L_i(\mu)$, es decir, los puntos de equilibrio del PC3C dependen todos del parámetro de masa del sistema. Los puntos $L_4(\mu)$ y $L_5(\mu)$ son de particular interés ya que, a diferencia de los otros tres, serán estables si $\mu \leq \mu_c$, donde $\mu_c$ es el parámetro crítico de la masa dado por (\ref{eq:mu_critica}). 

El análisis de eigenvalores alrededor de los puntos singulares nos da cómo es la estabilidad de dichos puntos. Sin embargo, es interesante ver las soluciones cerca de $\mu_c$ ya que nos permitirá ver directamente en el flujo el comportamiento de estabilidad de la partícula que visita a $L_4$ o $L_5$.

Para esto, se seguirá la filosofía de la sección \ref{sec:parameter_variation}, donde se agregará a las ecuaciones de movimiento la ecuación 
\begin{equation*}
 \dot{\mu} = 0.
\end{equation*}
Con esto, se hará un transporte en $\mu$, es decir, ésta se parametrizará como 
\begin{equation*}
 \mu = \mu_0 + \delta\mu
\end{equation*}   
donde $\delta\mu \in \pkr$ y, por tanto, $\mu \in \pkr$ también. Se introduce la notación $\phi_\mu(t;t_0,\xo)$ para referirse a este tipo de transporte. Tomando $\mu_0 = \mu_c \approx 0.0385$, se está en el límite entre la estabilidad y la divergencia alrededor de los puntos $L_4$ y $L_5$.

%FIGURA!
\begin{figure}
 \centering
 \includegraphics[width=0.8\linewidth]{flow_dmu}
 \caption{Espacio de configuraciones para $\phi_\mu$ con variaciones $\delta\mu = 0$ (azul), $\delta\mu = \delta\mu_{max}$ (naranja), y $\delta\mu = -\delta\mu_{max}$ (verde), donde $\delta\mu_{max} = \xi_{max}(\phi_\mu) \approx 0.00329$. Para el transporte se usó un jet de orden $n=18$ con condiciones iniciales $\left( L_{4_x}, L_{4_y} + \Delta y, 0, 0, \mu \right)$.}
 \label{fig:flow_dmu}
\end{figure}

Para el transporte, tomaremos como condición inicial a $\xo = \left( L_{4_x}, L_{4_y} + \Delta y, 0, 0, \mu \right)$ con $\Delta y = 7\times 10^{-4}$ para no estar exáctamente en el punto fijo. Las variaciones de $\delta\mu$ estarán acotadas por el tamaño máximo de vecindad de la expresión (\ref{eq:ximax}), es decir, $\lvert \delta\mu \rvert \leq  \xi_{max}(\phi_\mu) \approx 0.033$ en este caso. Se puede ver en la figura \ref{fig:flow_dmu} una integración a $40$ unidades temporales. Se observa cómo para $\delta \mu > 0$, las soluciones se alejan cada vez más de $L_4$, mientras que para $\delta \mu < 0$ se queda en cierta localidad. Una forma de observar este comportamiento de manera más clara es si medimos la separación de $\phi_\mu(t)$ respecto de $\xo$ para distintos valores de $\delta \mu$. La figura \ref{fig:remoteness_dmu} muestra esta separación en función del tiempo. Se puede observar que mientras mayor se la variación de $\mu$, mayor es la tendencia y mayor la amplitud de oscilación. 

%FIGURA! 
\begin{figure}
 \centering
 \includegraphics[width=0.8\linewidth]{remoteness_dmu}
 \caption{Diferencia $\Delta\phi_\mu(t) = \norm{ \phi_\mu(t) - \xo }$ del espacio de configuraciones para $\delta\mu \in \left\lbrace -\delta\mu_{max}, -2\delta\mu_{max}/3, -\delta\mu_{max}/3, 0, \delta\mu_{max}/3, 2\delta\mu_{max}/3, \delta\mu_{max}  \right\rbrace$.}
 \label{fig:remoteness_dmu}
\end{figure}

Una forma de ver realmente cuándo el flujo no diverge de la localidad de $L_4$ es si podemos aislar la tendencia de la amplitud de los ciclos. Para ésto, se puede usar un filtro de Hodrick-Prescott, que separa una serie de tiempo en tendencia y ciclo. Sea $y_t$ la variable de una serie de tiempo al tiempo $t$. Se asume que dicha serie está compuesta de una componente de tendencia $\tau_t$ más una componente ciclica $c_t$, es decir, $y_t = \tau_t + c_t$\footnote{Normalmente se incluye una componente de error $\eta_t$ que, en este caso, será absorbida por $c_t$.}. Así, uno puede encontrar la tendencia, como se plantea en \cite{Hodrick1997}, bajo el problema de minimización
\begin{equation}
 \min_{\tau} \left( \sum_{t=1}^T (y_t - \tau_t)^2 + \lambda \sum_{t=2}^{T-1} \left[ (\tau_{t+1} - \tau_t) - (\tau_t - \tau_{t-1}) \right]^2 \right)
 \label{eq:hp-filter}
\end{equation} 
con $\lambda$ un parámetro positivo que regula la variabilidad en la componente de tendencia de la serie.

Tomaremos la solución matricial lineal de \cite{Kim2004}  
\begin{equation}
 \mathbf{y}_t = \left( \lambda \mathbf{F} + \mathbf{I}_t \right) \tau_t
 \label{eq:hp_matrix}
\end{equation}  
al derivar (\ref{eq:hp-filter}) respecto a $\tau_t$, donde
\begin{align}
 \mathbf{F} = \left[ \begin{array}{cccccccccc}
 1  & -2 & 1  & 0  & \cdots &        &        & & \cdots & 0    \\
 -2 & 5  & -4 & 1  & 0      & \cdots &        & & \cdots & 0    \\
 1  & -4 & 6  & -4 & 1      & 0      & \cdots & & \cdots & 0    \\
 0  & 1  & -4 & 6  & -4 & 1 & 0      & \cdots &        & \vdots \\
 \vdots  & \ddots  &  &  &  &  &  &  &        & \ddots          \\
    &    &    &  0 &  1 &-4 &    6   & -4     &      1 & 0      \\
   &   &  & \cdots &  0 & 1 &   -4   &  6     &      4 & 1      \\
   &   &  &  &   \cdots & 0 &    1   &  -4    &      5 & 2      \\
 0 &\cdots&  &  &  & \cdots &    0   &   1    &     -2 & 1      \\
 \end{array} \right].
\end{align}

ya que ésta es fácilmente programable. De (\ref{eq:hp_matrix}), obtenemos que $\mathbf{\tau}_t = \left( \lambda \mathbf{F} + \mathbf{I}_t \right)^{-1} y_t$ y $c_t = y_t - \tau_t$. Aplicando este filtro en la figura \ref{fig:remoteness_dmu}, se obtiene el ciclo y la tendencia tal como se presenta en la figura \ref{fig:trendcycle_dmu}. En ésta, se observa cómo desde la primera variación negativa ($\delta\mu = -\delta\mu_{max}/3$) hacia valores más negativos, la tendencia decrece después de llegar a un máximo. Esto quiere decir que la distancia respecto al punto inicial no diverge, como se había predicho en el análisis de estabilidad lineal de (\ref{eq:stability_CR3BP}). Por otro lado, es interesante ver cómo la amplitud de las oscilaciones en la parte cíclico de $\Delta\phi_\mu$ son mayores entre mayor sea $\delta_\mu$.

%FIGURA!
\begin{figure}[h!]
\centering
\begin{subfigure}{0.49\textwidth}
	\centering
	\includegraphics[width = \textwidth]{trends_dmu}
\end{subfigure}
%
\begin{subfigure}{0.49\textwidth}
	\centering
	\includegraphics[width = \textwidth]{cycles_dmu}
\end{subfigure}
\caption{ Separación en tendencia (izquierda) y ciclo(derecha) de los valores de la figura \ref{fig:remoteness_dmu} usando un filtro de Hodrick-Prescott con $\lambda = 8000$.}
\label{fig:trendcycle_dmu}
\end{figure}

Hacer esta separación ayuda fuertemente al análisis de estabilidad de $\mu$ alrededor de $L_4$ y $L_5$ cuando $\mu = \mu_c \pm \delta\mu$, pero ¿Qué pasa en el caso donde $\mu \gg \mu_c$ como en el caso Tierra-Luna?

\subsection{Caso Tierra-Luna}
Dado que en la adimensionalización de las ecuaciones de movimiento el problema queda planteado únicamente en términos de $\mu$, basta ver que en el caso Tierra-Luna $M_T = 5.972 \times 10^{24} kg $, $M_L = 7.348 \times 10^{22} kg$ y, por tanto, $\mu_{TL} = 0.01215$. Notamos que $\mu_{TL} \gg \mu_c$ por lo que alrededor de $L_4$ y $L_5$ debería haber órbitas estables, aún cuando $\delta\mu_{max}$ sea grande.

%FIGURA!
\begin{figure}
 \centering
 \includegraphics[width=0.8\linewidth]{flow_tl}
 \caption{Espacio de configuraciones para $\phi_\mu$ con variaciones $\delta\mu = 0$ (azul), $\delta\mu = \delta\mu_{max}$ (naranja), y $\delta\mu = -\delta\mu_{max}$ (verde), donde $\delta\mu_{max} = \xi_{max}(\phi_\mu) \approx 0.0063$. Para el transporte se usó un jet de orden $n=18$ con condiciones iniciales $\left( L_{4_x}, L_{4_y} + \Delta y, 0, 0, \mu \right)$.}
 \label{fig:flow_tl}
\end{figure}

Primero, al hacer el transporte de jets a $40$ unidades temporales y ver el espacio de configuraciones como se muestra en la figura \ref{fig:flow_tl}, se observa cómo las tres trayectorias $\delta\mu = \left\lbrace -\delta\mu_{max}, 0, \delta\mu_{max} \right\rbrace$ quedan siempre orbitando alrededor de $L_4$, es decir, no divergen. Resulta que $\delta\mu_{max} = 0.0063 $, lo cual equivale a una variación del $52.14 \%$ respecto a $\mu_{TL}$. Para ponerlo en perspectiva, $\mu_{TL}$ equivale a que la masa primaria (la Tierra) es unas $81$ veces más grande que la secundaria (luna); $\mu_{TL} \pm \delta\mu_{max}$ equivale a que la masa primaria sea $53$ y $171$ veces, respectivamente. Análogo a la figura \ref{fig:remoteness_dmu}, se hace la gráfica de separación respecto a $\xo$ en la figura \ref{fig:remoteness_tl}. Aquí se puede observar que, aunque al principio (cerca de las $10$ unidades temporales) parecería empezar una tendencia similar al caso anterior, rápidamente pierda dicho orden y se vuelven oscilaciones completamente arbitrarias, sin tener una tendencia aparente y con amplitudes similares de ciclo.

%FIGURA!
\begin{figure}
 \centering
 \includegraphics[width=0.8\linewidth]{remoteness_tl}
 \caption{Diferencia $\Delta\phi_\mu(t) = \norm{ \phi_\mu(t) - \xo }$ del espacio de configuraciones para $\delta\mu \in \left\lbrace -\delta\mu_{max}, -2\delta\mu_{max}/3, -\delta\mu_{max}/3, 0, \delta\mu_{max}/3, 2\delta\mu_{max}/3, \delta\mu_{max}  \right\rbrace$.}
 \label{fig:remoteness_tl}
\end{figure}

De hecho, en la figura \ref{fig:trendcycle_tl} se muestran los gráficas al aplicar el filtro de Hodrick-Prescott. Al aplicar dicho filtro, ¡No se puede observar absolutamente nada! Ni la gráfica de ciclo ni la de tendencia tienen ninguna coherencia en realción a las variaciones del parámetro de masa. Esto significa que $\mu_{TL} + \delta\mu$ es, en efecto, estable para todo el conjunto de valores $\delta\mu$ evaluados alrededor de $L_4$ y $L_5$. Este es un caso donde el TJ es una herramienta natural para evaluar la estabilidad de éstos puntos lagrangianos y para ver cómo varían las trayectorias en función de los parámetros del problema-. Muchas otras exploraciones se pueden hacer con variación de parámetros en el problema de tres cuerpos. Sin embargo, este análisis basta para ver las capacidades del TJ cuando no son específicamente ``Jets'' los que transporta.

%FIGURA!
\begin{figure}[h!]
\centering
\begin{subfigure}{0.49\textwidth}
	\centering
	\includegraphics[width = \textwidth]{trends_tl}
\end{subfigure}
%
\begin{subfigure}{0.49\textwidth}
	\centering
	\includegraphics[width = \textwidth]{cycles_tl}
\end{subfigure}
\caption{ Separación en tendencia (izquierda) y ciclo(derecha) de los valores de la figura \ref{fig:remoteness_tl} usando un filtro de Hodrick-Prescott con $\lambda = 8000$.}
\label{fig:trendcycle_tl}
\end{figure}

\pagebreak
\section{Colisión de asteroides}

%%% Choro acerca de dos asteroides colisionando. %%% 

%%% Chorito acerca de cómo 4 masas es lo mismo que 3. (usando la masa de apohpis por ejemplo... %%%

%Hay en promedio $X$ objetos celestes orbitando a la Tierra 

%Hace 65 millones de años no existían métodos computacionales para determinar la trayectoria de las posibles colisiones de asteroides  con el planeta Tierra. ¿Qué sucedió?, los dinosaurios se extinguieron \cite{}. Hoy en día conocemos (o creemos conocer) las leyes que gobiernan la dinámica del sistema solar y los cuerpos que lo visitan así como métodos que resuelven dicha dinámica. Posiblemente no podamos detener el impacto, pero podemos intentar prepararnos para dicho suceso con varios años de anticipación. Por esto, es importante conocer el riesgo que tiene un objeto celeste de impactar con la Tierra, y el transporte de Jets es una muy buena herramienta para analizar dicha posibilidad. 

%con velocidad $\mathbf{v}_{a_1}(t_0) + \delta\mathbf{v}_{a_1}(t_0)$

Sean $a_1$ y $a_2$ dos asteroides con la misma energía de Jacobi alrededor de la masa primaria mayor. Inicialmente, éstos se encuentran a $\mathbf{x}_{a_1}(t_0) + \delta\mathbf{x}_{a_1}(t_0)$  y $\mathbf{x}_{a_2}(t_0) + \delta\mathbf{x}_{a_2}(t_0)$, respectivamente, donde $\delta\mathbf{x}_{a_i} \in \pkk{n}{2}$. La variación en cada condición representa el error de medición de éstos. Por tomar una referencia, tomemos la incertidumbre en el orden de los datos del asteroide Apophis en los años $2012 - 2028$ \cite{Desmars2013}, donde definimos $\Delta_{max} := 350 $ km como la incertidumbre máxima para los asteroides.

La figura \ref{fig:asteroid_collision_nominal_integration} muestra la trayectoria nominal de $a_1$ y $a_2$ para condiciones iniciales con energía $C_J = -1.81252 < C_J(L_1)$. En esta integración, la mínima distancia entre ambos cuerpos es de $0.00348$ que representa unos $1341$ kilómetros. Sin embargo, aunque para esta condición particular los asteroides no chocan, es posible que sí lo hagan dentro de una incertidumbre de radio $\Delta_{max}$. Esto se puede hacer con el transporte de jets de la siguiente manera: 

%FIGURA!
\begin{figure}
 \centering
 \includegraphics[width=0.7\linewidth]{asteroid_collision_nominal_integration}
 \caption{Integración nominal a $T=10 \approx $ unidades temporales, donde $\mathbf{q}_{a_1}(t_0) = \left( -0.373098, -0.0804321, 1.21995, \dot{y}_{a_1} \left( x_{a_1}(t_0), y_{a_1}(t_0), \dot{x}_{a_1}(t_0), C_J \right) \right)$ 
 y $\mathbf{q}_{a_2}(t_0) = \left( -0.27324, -0.307896, -0.307576, \dot{y}_{a_2} \left( x_{a_2}(t_0), y_{a_2}(t_0), \dot{x}_{a_2}(t_0), C_J \right) \right)$. El círculo naranja representa a $M_1$, y los círculos verde y gris a $\mathbf{q}_{a_1}(t_0)$ y $\mathbf{q}_{a_2}(t_0)$, respectivamente. La zona de mayor acercamiento se marca con cruces, donde en ésta, la distancia entre los asteroides es de $0.00348 \approx 1341$ km.}
 \label{fig:asteroid_collision_nominal_integration}
\end{figure}

\begin{itemize}
 \item Hacer una integración nominal de dos asteroides con energías similares o que se sepa que pueden tener riesgo de colisión.
 
 \item Encontrar las coordenadas $\mathbf{q}_{a_i}^{(col)}$ donde la distancia entre $a_1$ y $a_2$ es mínima.
 
 \item Hacer TJ alrededor de las condiciones encontrada en el punto anterior.
 
 \item Evaluar una distribución de $N$ puntos en la bola $\delta\mathbf{x}_i \leq \Delta_{max}$ alrededor de $\mathbf{q}_{a_i}^{(col)}$ para ambos asteroides y guardarla.
 
 \item Definir una distancia $D_{col}$ para la cual los asteroides chocarían y determinar qué puntos de la distribución quedan debajo de ésta.
 
 \item Determinar la probabilidad de impacto $\mathcal{P}$ como la cantidad de veces puntos que quedan debajo de $D_{col}$ respecto al número de trayectorias totales $N^2$; así
 \begin{equation} 
 \mathcal{P} = \frac{\norm{\mathbf{x}_{a_1} - \mathbf{x}_{a_2} }    \leq D_{col} }{N^2}.
 \end{equation}
\end{itemize}

%FIGURA!
\begin{figure}
 \centering
 \includegraphics[width=0.7\linewidth]{asteroid_collision_distribution}
 \caption{Transporte de jets de orden $4$ alrededor de $\mathbf{q}_{a_i}^{(col)}$ (cruces en la figura) donde los cúmulos son la distribución de radio $\delta\mathbf{x}_{a_i} \leq 350$ km evaluadas en el polinomio resultante del transporte. Para el método sólo se toman los cúmulos donde se encuentra $\mathbf{q}_{a_i}^{(col)}$, pero la figura busca ilustrar una sección de la trayectoria de los asteroides.}
 \label{fig:asteroid_collision_distribution}
\end{figure}

La figura \ref{fig:asteroid_collision_distribution} muestra una distribución normal de $7000$ puntos\footnote{$7000$ ya convergen a una probabilidad cuyas cifras significativas no afectan el redondeo de la tabla \ref{table:collision_table}.} evaluados en el transporte alrededor de $\mathbf{q}_{a_i}^{(col)}$ para $i = \{1,2\}$. La probabilidad de impacto dependerá del tamaño que tengan los asteroides, la cual se presenta en la tabla \ref{table:collision_table}. Se presenta en la figura \ref{fig:asteroid_collision_histogram} la distribución de distancias entre ambos asteroides en la zona de riesgo de colisión. Aún cuando $D_{col}$ e incluso $\Delta_{max}$ quedan debajo del escenario más posible en la distribución, la probabilidad de impacto es suficientemente grande para no ser ignorada en los asteroides de mayor dimensión.

%TABLA!
\begin{table}[h!]
\centering
\begin{tabular}{c|c|c}
\toprule
\textbf{$ D_{col}$ [ km ]} & \textbf{$\mathcal{P}$ [ $\%$ ]} & \textbf{Colisiones [ $ \# $ ]} \\ \cmidrule(l){1-3} 
\textbf{$0.5$} &   $3.7 \times 10^{-5}$   & $18$          \\
\textbf{$5$}   &   $0.0048$               & $2356$        \\
\textbf{$30$}  &   $0.172$                & $84,609$     \\
\textbf{$150$} &   $4.357$                & $2,135,173$   \\ \bottomrule 
\end{tabular}
\caption{Número de colisiones y riesgo de choque para asteroides de distintas dimensiones.}
\label{table:collision_table}
\end{table}

%FIGURA!
\begin{figure}
 \centering
 \includegraphics[width=0.7\linewidth]{asteroid_collision_histogram}
 \caption{Distribución normalizada de distancias entre $a_1$ y $a_2$ cerca del punto de posible dadas $7000$ variaciones iniciales alrededor de $\mathbf{q}_{a_1}^{col}$ y $\mathbf{q}_{a_2}^{col}$. Las líneas verticales muestran los valores usados en la tabla \ref{table:collision_table}.}
 \label{fig:asteroid_collision_histogram}
\end{figure}


Algo relevante al hacer transporte de jets es la presición de éste. Se encuentra que $\xi_{max}(\phi_{a_1}) = 0.0022 \approx 865 \ km \gg \Delta_{max}$\footnote{$\xi_{max}(\phi_{a_2})$ es prácticamente igual a $\xi_{max}(\phi_{a_1})$.}, por lo que se tiene bastante confianza en la precisión de las evaluaciones de la distribución. De hecho, al tomar una serie de variaciones $\norm{\delta\mathbf{x}_{a_i}} = \Delta_{max}$ y hacer la integración nominal de éstas, se encuentra que el error promedio respecto a éstas es del orden de $10^{-11} \approx 4 mm$, o sea, nada. Ésto se muestra en la figura \ref{fig:asteroid_jt_vs_nominal}.

%FIGURA!
\begin{figure}
 \centering
 \includegraphics[width=0.7\linewidth]{asteroid_jt_vs_nominal}
 \caption{Promedio de la distancia entre la integración nominal y el transporte de jets para $500$ condiciones iniciales con variación $\norm{\delta\mathbf{x}_{a_1}} = \norm{\delta\mathbf{x}_{a_1}} = \Delta_{max}$.}
 \label{fig:asteroid_jt_vs_nominal}
\end{figure}

Así, ilustramos el método de posible colisión entre asteroides, donde se estableció la probabilidad de impacto para diferentes zonas de colisión. El transporte de jets permitió encontrar dichas probabilidades ya que en éste, simplemente hubo que evaluar la distribución de variaciones iniciales en un radio dado. Éste método es generalizable a cualquier sistema donde exista alguna incertidumbre en las condiciones iniciales. Sin embargo, hay que tener claro que el transporte de jets no es muy preciso para integraciones muy largas ni para vecindades muy grandes. En este ejemplo, el radio de las variaciones eran $0.46$ veces más pequeñas que $\xi_{max}$. Además, se hizo la integración de jets cerca del punto de posible colisión para tener tiempo de integración cortos. 

%Mencionar adimensionalización del tiempo para ver cuánto implica 1 unidad temporal.

\section{Simplecticidad del PC3C}
Se construyeron en el capítulo \ref{chap:CR3BP} los potenciales (\ref{eq:3body_potential}) y sus respectivas energías cinéticas (\ref{eq:3body_kinetic}) del problema general de tres cuerpos. Éstas nos permiten definir al hamiltoniano
\begin{equation}
 H(\mathbf{r}_1, \mathbf{r}_2, \mathbf{r}_3, \mathbf{p}_1, \mathbf{p}_2, \mathbf{p}_3) = \sum_{i=1}^3 h_i(r_{i,j}, r_{i,k}, \mathbf{p}_i)
 \label{eq:3BP_hamiltonian}
\end{equation} 
donde $\mathbf{p}_i = M_i \dot{\mathbf{r}}_i$ es el momento conjugado y
\begin{equation}
 h_i(r_{i,j}, r_{i,k}, \mathbf{p}_i) = \frac{1}{2 M_i} p_i^2 + U_{M_i}(r_{i,j}, r_{i,k})
 \label{eq:individual_hamiltonian}
\end{equation}
los hamiltonianos para cada una de los cuerpos. Notemos que (\ref{eq:3BP_hamiltonian}) no depende del tiempo y, por lo tanto, es un sistema conservativo. 

Cuando en la sección \ref{sec:R3BP} se supuso que la masa menor no inlfuye en la dinámica de las masas primarias, bastó con hacer una rotación con velocidad angular $\Omega$ para encontrar las ecuaciones de movimiento de $m_3$. Ésto fija las masas primarias, cancela sus hamiltonianos ($  U_{M_2} = - U_{M_1}$) y modifica el hamiltoniano para la masa menor. Dicha rotación se puede realizar bajo el cambio de coordenadas
\begin{align*}
 x &= X\cos (\Omega t) - Y \sin (\Omega t) \\
 y &= X\sin (\Omega t) - Y \cos (\Omega t) 
\end{align*} 
con $(x,y)$ y $(X,Y)$ las nuevas y viejas coordenadas para $m_3$, respectivamente, tal como se muestra en la figura \ref{fig:FIGURA!}.

%FIGURA!

Para obtener el hamiltoniano modificado, que en alguna literatura se llama kamiltoniano \cite{Goldstein2007, Johns2011}, es necesario ver cómo se modifica el momento conjugado en el nuevo sistema donde, para esto, se debe conservar el principio de mínima acción

\begin{align}
 \delta \int_{t_0}^t \left( \dot{\mathbf{R}} \cdot \mathbf{P} - H(\mathbf{R}, \mathbf{P}, \tau) \right) d\tau &= 0 \nonumber \\
\implies \delta \int_{t_0}^t \left( \dot{\mathbf{r}} \cdot \mathbf{p} - K(\mathbf{r}, \mathbf{p}, \tau) \right) d\tau &= 0
 \label{eq:kamiltonian_condition}
\end{align}
en ambos ejes, lo cual sucede si
\begin{equation}
 \dot{\mathbf{R}} \cdot \mathbf{P} - H(\mathbf{R}, \mathbf{P}) = \dot{\mathbf{r}} \cdot \mathbf{p} - K(\mathbf{r}, \mathbf{p}, t) + \frac{dG(\mathbf{R}, \mathbf{p}, t)}{dt}.
\end{equation}
Aquí $\mathbf{P} = (P_x,P_y)$, $\mathbf{R} = (X,Y)$, $\mathbf{p} = (p_x,p_y)$, $\mathbf{r} = (x,y)$, $K$ es el kamiltoniano y $G = - \mathbf{r} \cdot \mathbf{p} + G_2(\mathbf{R}, \mathbf{p}, t)$ es una función que genera una transformación canónica entre ambos sistemas, también conocida como función generatriz de tipo 2 \cite{CanonicalTransformations}. Notemos que como 
\begin{align*}
 \frac{dG}{dt} =  - \dot{\mathbf{r}}\cdot \mathbf{p} - \mathbf{r}\cdot \dot{\mathbf{p}} + \pder{G_2}{\mathbf{R}}\dot{\mathbf{R}} + \pder{G_2}{\mathbf{p}}\dot{\mathbf{p}} + \pder{G_2}{t}
\end{align*}
entonces
\begin{align}
 \mathbf{P} &= \pder{G_2}{\mathbf{R}} \nonumber \\
 \mathbf{r} &= \pder{G_2}{\mathbf{p}} \nonumber \\
 \therefore K(\mathbf{r}, \mathbf{p}) &= H(\mathbf{r}, \mathbf{p}) + \pder{G_2}{t} (\mathbf{r}, \mathbf{p}).
 \label{eq:kamiltonian_formulae}
\end{align}

Definimos $G_2$ simplemente como
\begin{equation}
 G_2(\mathbf{R}, \mathbf{p}, t) = p_x \underbrace{\left( X \cos (\Omega t) - Y \sin (\Omega t) \right) }_{x} + p_x (\underbrace{ \left( X \sin (\Omega t) + Y \cos (\Omega t) \right)}_{y}
 \label{eq:generating_function}
\end{equation}

y, con (\ref{eq:kamiltonian_formulae}), permite formular explícitamente al kamiltoniano como
\begin{equation*}
 K(\mathbf{r}, \mathbf{p}) = \underbrace{ \frac{1}{2M_3}\left( p_x^2 + p_y^2 \right) - G \frac{M_1}{r_{1,3}} - G \frac{M_2}{r_{2,3}}}_{H(\mathbf{r}, \mathbf{p}) } + \underbrace{\Omega\left( p_y x - p_x y \right) }_{\pder{G}{t}(\mathbf{r}, \mathbf{p})}
\end{equation*}

\begin{equation}
 \therefore K = \frac{1}{2M_3} \left( (p_x - \Omega y)^2 + (p_y + \Omega x)^2 \right) - \left(G \frac{M_1}{r_{1,3}} + G \frac{M_2}{r_{2,3}} + \frac{\Omega^2}{2}\left( x^2 + y^2 \right) \right).
 \label{eq:kamiltonian}
\end{equation}

Notemos que $K = k_3$, así que la partícula menor define al hamiltoniano del sistema en los nuevos ejes. Así, para probar la simplecticidad del PC3C, basta con hacer las transformaciones 
\begin{align}
p_x &\to p_x - \Omega y \\
p_y &\to p_y + \Omega x
\label{eq:momentum_transformation}
\end{align} 
y aplicar la relación (\ref{eq:sympletic_flow}).

%FIGURA!
\begin{figure}[h!]
\centering
\begin{subfigure}{0.49\textwidth}
	\centering
	\includegraphics[width = \textwidth]{non_symplectic_L1}
\end{subfigure}
%
\begin{subfigure}{0.49\textwidth}
	\centering
	\includegraphics[width = \textwidth]{non_symplectic_L4}
\end{subfigure}
\caption{ $\varsigma(t)$ para $\phi(t)$ sin modificar el flujo. Las condiciones iniciales son $\xo = \left( \left( L_{1_x}, 0.001, 0, 0 \right)^T + \delta \xi \right) $ (izquierda) y $ \xo = \left( \left( L_{4_x}, L_{4_y} + 0.001, 0, 0 \right)^T + \delta \xi \right) $ (derecha) para $\delta \xi$ un polinomio de orden $3$.}
\label{fig:non_symplectic_L4_L1}
\end{figure}

Si uno no hace dichas transformaciones, parecerá que el problema no conserva la forma simpléctica, tal como en el ingenuo intento que se muestra en la figura \ref{fig:non_symplectic_L4_L1} para diferentes condiciones iniciales. Para obtener $\varsigma$ se utiliza la ecuación (\ref{eq:scalar_symplecticity}) que, como se discute en la sección \ref{sec:formas-simplecticas}, opera de manera muy natural con el TJ, ya que la parametrización de las vecindades del espacio fase permite computar al jacobiano en cada punto de manera directa.

La figura \ref{fig:symplectic_L4_L1}, en cambio, muestra la conservación simpléctica del PC3C bajo la transformación (\ref{eq:momentum_transformation}), donde se observa cómo $\varsigma(t)$ se mantiene constante durante toda la inegración. En ésta, se tomaron las mismas dos condiciones iniciales que en la figura \ref{fig:non_symplectic_L4_L1}; una cerca de $L_4$, que es una trayectoría estable, y otra cerca de $L_1$, que orbita alrededor de la masa primaria menor. Se puede observar cómo en la condición más estable la conservación de la simplecticidad varía en unos dos órdenes de magnitud menos que cerca de $L_1$. Sin emabrgo, ambas tienen variaciones menor a $10^{-10}$ en cada punto de la trayectoria, porl o que podemos decir que, en efecto, el sistema es simpléctico. Esta es una aplicación muy directa para el TJ que aprovecha su estructura paramétrica, de la cual se pueden obetener el jacobiano o incluso variaciones de órdenes mayores. 

%FIGURA!
\begin{figure}[h!]
\centering
\begin{subfigure}{0.49\textwidth}
	\centering
	\includegraphics[width = \textwidth]{symplectic_L1}
\end{subfigure}
%
\begin{subfigure}{0.49\textwidth}
	\centering
	\includegraphics[width = \textwidth]{symplectic_L4}
\end{subfigure}
\caption{$\varsigma(t)$ para $\phi(t)$ con las transformaciones (\ref{eq:momentum_transformation}). Las condiciones iniciales son $\xo = \left( \left( L_{1_x}, 0.001, 0, 0 \right)^T + \delta \xi \right) $ (izquierda) y $ \xo = \left( \left( L_{4_x}, L_{4_y} + 0.001, 0, 0 \right)^T + \delta \xi \right) $ (derecha) para $\delta \xi$ un polinomio de orden $3$.}
\label{fig:symplectic_L4_L1}
\end{figure}

%Valdrá la pena hacer un análisis simpléctico de mayor dimensión?

\section{Tamaño máximo de vecindades como indicador de sensibilidad de las condiciones iniciales}

.