En esta tesis se desarrolla el concepto del transporte de jets motivado por los estudios de Daniel Pérez-Palau en \cite{Perez2013, Perez2015} y se aplica en casos particulares del problema circular de tres cuerpos. El transporte de jets integra el flujo de una ecuación diferencial alrededor de una vecindad $\U$ en lugar de una única  condición inicial $\xo$. Para hacerlo, se parametriza a $\U$ con un polinomio de $k$ variables, donde $k$ es la dimensión del espacio y se aplica a cualquier método de integración numérica, que en este trabajo será el método de Taylor por cuestiones de precisión. Como el método de Taylor depende de $\mathbf{x}_n$ para obtener a $\mathbf{x}_{n+1}$, el transporte de jets dependerá de $\mathcal{U}_{\mathbf{x}_n}$ para obtener $\mathcal{U}_{\mathbf{x}_{n+1}}$ y, como $\mathcal{U}_{\mathbf{x}_n}$ es un polinomio, operar con éste requiere de la implementación de un álgebra polinomial, que estará guiada por la construcción de Haro en \cite{Haro2009}. Las soluciones para las distintas variaciones respecto a $\mathcal{U}_{\mathbf{x}_n}$ se obtienen simplemente al evaluar el polinomio en cuestión. Al tener parametrizada toda una vecindad de $\xo$, se proponen en la tesis varios indicadores y algoritmos que aprovechan dicha parametrización.

En el problema circular de tres cuerpos se aplican dichos indicadores y algoritmos desarrollados con el transporte de jets para encontrar la probabilidad de colisión entre asteroides de diferente radio en el sistema Tierra-Luna, probar, en un sentido numérico, que el integrador preserva la simplecticidad del sistema, estudiar la sensibilidad de condiciones iniciales y analizar la variación del parámetro de masa alrededor de puntos de equilibrio del espacio de configuraciones.  

Adicionalmente, se evalúa la eficiencia, la presición y los tiempos involucrados para los algoritmos desarrollados, los cuales están disponibles en \href{https://github.com/blas-ko/tesis}{https://github.com/blas-ko/tesis} y se programaron en el lenguaje Julia, haciendo uso de los paquetes \textsf{TaylorSeries} \cite{TaylorSeries} y \textsf{TaylorIntegration} \cite{TaylorIntegration}. 
