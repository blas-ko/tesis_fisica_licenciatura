%%%% INTRODUCCIÓN A LAS ECUACIONES DE  BOSE-EINSTEIN, MOTIVACIÓN DE ESTAS Y DESARROLLO DE LAS ECUACIONES. ¿CUÁNDO SE DE EL CONDENDSADO? ¿QUÉ APROXMACIONES VAMOS A TOMAR PARA ESTE PROBLEMA? 

Tomando el estado base del condensado, que es equivalente al estado de temperatura cero, es interesante estudiar la dinámica de los estados de spín de los bosones en el sistema. Al tener estos un spin $s = 1$, cada átomo tendrá una posible proyección de spin $m_s \in \left\lbrace +,0,- \right\rbrace$, por tanto, la dinámica de estos y el espacio fase generado en el la capa de energía cero nos dará propiedades cuánticas sobre dicho sistema. 

el hamiltoniano general en sus segunda cuantización para un grupo de bosones que interactuán en un sistema con un potencial externo $V(x)$ es ( $ \hbar = 1$ )

\begin{equation}
\begin{split}
\hat{H} & = \sum_{\alpha} \int \afc{\alpha}(x) \left( -\frac{\nabla^2}{2m} + V_{Trap}(x) \right) \afa{\alpha}(x) d^3x \\ 
& + \sum_{\alpha,\beta, \mu, \nu} \int d^3x_1 \int d^3x_2  	\afc{\alpha}(x_1) \afc{\beta}(x_2) \hat{U}(x_1,x_2) \afa{\mu}(x_1) \afa{\nu}(x_2)
\end{split}
\label{gen_ham}
\end{equation}

%%% OJO: Este hamiltoniano NO presupone un mínimo de energía, es un hamiltoniano general, por tanto el estado de mínima energía deberá ser encontrado a posteriori del desarrollo.

donde $\afc{\alpha}$ es el operador de campo de aniquilación para el estado $\alpha = \ket{s=1,m_s=\alpha}$, $V_{Trap}$, por otro lado, es el potencial que atrapa a los bosones, que, como se discute en \cite{trap}, es un sistema magnético (...).
$\hat{U}$ es el potencial de interacción entre las partículas del sistema. Es pertinente pensar que los bosones interactúan únicamente al colisionar, describiendo dicha interacción con

\begin{equation}
\hat{U}(x_1,x_2) = \delta{(x_1 - x_2)} \sum_{S=0}^2 g_S \sum_{M_S=-S}^S \ket{S,M_S}\bra{S,M_S}
\label{pseudopot}
\end{equation}

%%% hablar sobre los coeficientes g_S, el operador de proyeccción y qué significa el ket-bra. 
%%% Intentar hacer clara la analogía de porque se usa el s-wave scattering y hacer referencia a estos papers

Seguiremos la estructura planteada por \cite{law98} para estudiar a fondo \ref{gen_ham}.  

%%% Desarrollar o plantear el desarrollo del ket-bra via los coeficientes de clebsch-gordon. 
%%% Obtener los hamiltonianos simetricos (H_s) y no simétrico (H_a).
%%% Hablar de la aproximación phi_k(x) ≈ phi(x)“ cuando Hs/Ha >> 1 y dar ejemplos reales donde sí pase esto. 
%%% Usar las ecuaciones de Gross-Pitaevskii para obtener la parte simétrica del hamiltoniano en términos de N.
%%% Estructuras similares en nonlinear wave-mixing processes in cavity QED [ref]. Así, se puede hacer una estructura de momentos angulares definidas como L_, L+, Lz usadas en [refs] y trabajar con Ha para obtenerla en términos de L^2.

%%% Discusión general de las ecuaciones encontradas si es necesario

%%%%%%%%%%%%%%%%%%%%%%%% MEAN-FIELD APPROXIMATION %%%%%%%%%%%%%%%%%%%%%%%%%



